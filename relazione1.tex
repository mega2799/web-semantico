% \documentclass{article}
% \usepackage{graphicx} % Required for inserting images

% \title{WS-proj 1}
% \author{matteo.santoro6 }
% \date{November 2025}

% \begin{document}

% \maketitle

% \section{Introduction}

% \end{document}


\documentclass[a4paper,12pt]{article}

% --- Pacchetti di base ---
\usepackage[utf8]{inputenc}
\usepackage[T1]{fontenc}
\usepackage[italian]{babel}
\usepackage{geometry}
\usepackage{hyperref}
\usepackage{graphicx}
\usepackage{setspace}
\usepackage{listings}
\usepackage{xcolor}
\usepackage{tikz}
% --- Impostazioni ---
\geometry{margin=2.5cm}
\setstretch{1.2}

% --- Stile per codice RDF/OWL ---
\lstdefinelanguage{turtle}{
  keywords={@prefix,a},
  sensitive=false,
  morecomment=[l]{#},
  morestring=[b]"
}
\lstset{
  basicstyle=\ttfamily\small,
  breaklines=true,
  frame=single,
  backgroundcolor=\color{gray!5},
  keywordstyle=\color{blue},
  commentstyle=\color{gray!70},
  stringstyle=\color{orange!90!black}
}

\begin{document}

% --- Titolo ---
\begin{titlepage}
  \centering
  \vspace*{3cm}
  {\Large Università di Bologna}\\[0.3cm]
  {\large Corso di Web Semantico}\\[2cm]
  {\LARGE \textbf{Analisi dell’Ontologia della Birra}}\\[0.5cm]
  {\large Caso di studio: Beer Ontology}\\[1.5cm]
  {\large Studente: Matteo Santoro}\\[0.3cm]
  {\large Matricola: 1086420}\\[3cm]
  {\large Anno Accademico 2024–2025}
\end{titlepage}

\tableofcontents
\newpage

% --- 1. Introduzione ---
\section{Introduzione}
Il Web Semantico mira a rendere i contenuti del Web interpretabili non solo dagli esseri umani, ma anche dalle macchine. 
Le ontologie svolgono un ruolo fondamentale in questo contesto, fornendo una rappresentazione formale della conoscenza tramite concetti, relazioni e proprietà.

In questo elaborato viene analizzata la \textbf{Beer Ontology}\footnote{\url{https://rdf.ag/o/beer-en.html}}, un esempio didattico di ontologia sviluppata per descrivere il dominio della birra, dei birrifici e delle loro caratteristiche. 
L’obiettivo è comprendere la struttura, gli standard del Web Semantico impiegati e le potenzialità offerte.

% --- 2. Contesto generale ---
\section{Contesto generale}
La \textit{Beer Ontology} fa parte del più ampio insieme di ontologie pubblicate sul portale \href{https://rdf.ag}{rdf.ag}, una collezione di risorse RDF e OWL sviluppate per scopi accademici e didattici nel contesto del Web Semantico. 
Il portale fornisce un insieme di vocabolari interconnessi, progettati per illustrare le principali tecniche di modellazione della conoscenza, l’uso dei namespace e l’integrazione semantica tra domini differenti.

In questo ecosistema, la \textit{Beer Ontology} rappresenta un dominio specifico — quello della produzione e classificazione della birra — e viene utilizzata come esempio pratico per dimostrare come concetti del mondo reale possano essere formalizzati attraverso linguaggi semantici standard come RDF e OWL. 
Essa descrive entità come birrifici, stili di birra, ingredienti, caratteristiche organolettiche e relazioni di produzione, permettendo così di esplorare il potenziale del Web Semantico su un dominio concreto ma facilmente comprensibile.

Le finalità principali dell’ontologia e del progetto di cui fa parte sono:
\begin{itemize}
  \item fornire un caso di studio accessibile per l’apprendimento della modellazione semantica;
  \item mostrare l’interoperabilità dei dati attraverso l’uso di namespace condivisi e vocabolari riutilizzabili;
  \item consentire l’applicazione di tecniche di inferenza automatica tramite motori OWL, per derivare conoscenza implicita;
  \item favorire la sperimentazione con strumenti del Web Semantico come Protégé e SPARQL.
\end{itemize}

Grazie alla sua semplicità e chiarezza concettuale, la \textit{Beer Ontology} si presta come punto di partenza ideale per comprendere i principi della rappresentazione della conoscenza e le buone pratiche di progettazione ontologica promosse dal Web Semantico.


% % --- 3. Strumenti e linguaggi del Web Semantico utilizzati ---
% \section{Strumenti e linguaggi del Web Semantico}
% L’ontologia impiega diversi standard fondamentali del Web Semantico:
% \begin{description}
%   \item[RDF:] per rappresentare le informazioni sotto forma di triple soggetto–predicato–oggetto;
%   \item[RDFS:] per definire le classi e le relazioni di base;
%   \item[OWL:] per estendere la semantica e consentire ragionamenti più complessi;
%   \item[SPARQL:] per interrogare i dati RDF;
%   \item[IRI e namespace:] per garantire l’univocità dei concetti.
% \end{description}

% Esempio di frammento RDF in formato Turtle:

% \begin{lstlisting}[language=turtle, caption={Esempio RDF in Turtle}]
% @prefix beer: <http://purl.org/ontology/beer/> .

% beer:Guinness a beer:Stout ;
%     beer:brewedBy beer:GuinnessBrewery ;
%     beer:hasAlcoholContent "4.2"^^xsd:float .
% \end{lstlisting}

\section{Struttura concettuale dell’ontologia}
La Beer Ontology organizza la conoscenza relativa al dominio della birra in modo modulare e pragmatico. 
La progettazione privilegia chiarezza d'uso e riutilizzabilità dei vocaboli, rendendo esplicite molte scelte anche quando l'inferenza potrebbe derivarle automaticamente. 
L'ontologia si articola principalmente attorno a tre costrutti strutturali maggiori: \emph{Beer Styles} (stili di birra), \emph{Beer Organizations} (organizzazioni produttrici) e \emph{Packaging} (imballaggio/prodotto commerciale).\footnote{Vedi la descrizione completa su \url{https://rdf.ag/o/beer-en.html\#organization}.}

\subsection{Classi principali}
Le classi di alto livello definite nell’ontologia includono, tra le altre:
\begin{itemize}
  \item \textbf{Beer} — rappresenta la sostanza/alimento fermentato (il prodotto “birra” nella sua essenza);
  \item \textbf{Brewery} — l'organizzazione o il luogo fisico dove la birra viene prodotta (birrificio);
  \item \textbf{BeerStyle} — tassonomia degli stili di birra (es. \emph{Lager}, \emph{Ale}, \emph{Stout});
  % \item \textbf{Ingredient} — elementi costitutivi della birra (malto, luppolo, lievito, acqua, ecc.);
  % \item \textbf{Country} — entità geografica/paese di origine;
  \item \textbf{BeerProduct} / \textbf{BeverageProduct} — il prodotto commerciale confezionato (distinto dalla sostanza pura).
\end{itemize}

\subsection{Tre costrutti strutturali}
\paragraph{Beer Styles.}
Gli \emph{stili} sono definiti in base a ingredienti e processi di produzione. 
Questa scelta serve a evitare confusione tra stili “scientifici/tecnici” e termini di marketing. 
Gli stili sono modellati come sottoclassi specializzate (ad esempio \texttt{BrownAle} come specializzazione di una classe più generale), permettendo inferenze di tipo gerarchico. 
Alcuni punti salienti:
\begin{itemize}
  \item \textit{Brown Ale} — viene specificato come termine descrittivo: il colore dipende dal tipo e dalla quantità di malto, più che dal nome stesso.
  \item \textit{English vs British} — l’ontologia usa spesso \textit{English Style} come naming (es. foaf:name) per convenzione terminologica.
  \item \textit{Lagers vs Ales} — la distinzione è basata su caratteristiche biologiche/di processo (fermentazione con lieviti “bottom” vs “top”) anziché su luoghi geografici.
  \item \textit{Non-Beers} — bevande affini come sidro e idromele sono incluse per comodità d’uso ma classificate come non-beer (non derivano dalla fermentazione dei malti).
\end{itemize}

\paragraph{Beer Making Organizations.}
La classe \texttt{Brewery} rappresenta organizzazioni o luoghi fisici di produzione; qui sono modellati attributi come paese, nome commerciale, e relazioni con le birre prodotte tramite proprietà oggetto quali \texttt{brewedBy} (Beer → Brewery).

\paragraph{Packaging (imballaggio e prodotto commerciale).}
L’ontologia distingue in modo esplicito la \emph{birra} (la sostanza) dal \emph{beer product} (unità commerciale confezionata). Il modello del packaging cerca un compromesso tra completezza descrittiva e semplicità operativa:
\begin{itemize}
  \item sono disponibili misurazioni multiple (sistema imperiale e metrico) per le stesse dimensioni; le unità vengono identificate con standard quali UN/ECE Recommendation No.21 e il vocabolario \texttt{quot};
  \item la rappresentazione delle quantità commerciali utilizza concetti mutuati da GS1 e \texttt{schema.org}; per misure nominali o ingegneristiche si porge attenzione all'interoperabilità con ontologie per misure (es. SOSA/SSN);
  \item sono modellati alcuni contenitori tipici (keg, lattine da 355\,ml, lattine da 473\,ml). Materiale del contenitore e dettagli di sealing non sono ancora modellati in questa versione.
  \item \textit{Compound packaging} (es. six-packs, case da 24) è prevista per sviluppi futuri.
\end{itemize}

\subsection{Proprietà principali (concettuali)}
La Beer Ontology descrive le principali relazioni tra le entità del dominio, senza definire nomi concreti di proprietà. Tra i concetti principali troviamo:

\begin{itemize}
  \item Collegamento tra \textbf{Beer} e il birrificio che la produce (\textit{Brewery}) — concettualmente rappresenta chi produce la birra.
  \item Associazione tra \textbf{Beer} e il suo stile (\textit{BeerStyle}) — utile per classificare le birre secondo ingredienti e processo di fermentazione.
  \item Collegamento tra \textbf{Beer} e gli ingredienti principali (\textit{Ingredient}) — consente di modellare la composizione della birra.
  \item Informazioni sul contenuto alcolico e altre caratteristiche organolettiche — descritte come valori numerici o qualitativi.
  \item Collegamento tra \textbf{Beer} e il suo packaging commerciale (\textit{BeerProduct}) — rappresenta il contenitore e le quantità nominali.
\end{itemize}

\noindent
\textbf{Nota:} queste relazioni vanno considerate come concettuali e descrittive. La versione OWL pubblica della Beer Ontology contiene le classi e i concetti principali, ma non fornisce URI concreti di proprietà oggetto o dati. Queste informazioni servono come guida alla modellazione semantica e all’integrazione con vocabolari esterni (es. \texttt{schema.org}, \texttt{GS1}, \texttt{SOSA/SSN}).





\newpage
\begin{figure}[!h]
    \caption{Tipologie di birra}
    \centering
    \includegraphics[width=105mm]{beer_tpes.png}
    \label{fig:label}
\end{figure}
\begin{figure}[!h]
    \caption{Birrifici}
    \centering
    \includegraphics[width=105mm]{birrifici.png}
    \label{fig:label}
\end{figure}
\newpage

\subsection{Distinzione tra Beer e BeerProduct}
Una scelta modelluale importante è la distinzione concettuale tra la \texttt{Beer} (sostanza/entità organolettica) e il \texttt{BeerProduct} (unità commerciale confezionata). 
Questa separazione permette di:
\begin{itemize}
  \item associare a \texttt{BeerProduct} metadati commerciali (branding, prezzo, codice GS1, inventario) senza confondere tali informazioni con le proprietà organolettiche della bevanda;
  \item interoperare con vocabolari e standard di commercio elettronico (es. GS1, \texttt{schema.org}) per scopi pratici come catalogazione o commercio.
\end{itemize}


\subsection{Scelte di modellazione e implicazioni pratiche}
\begin{itemize}
  \item \textbf{Esplicitazione versus inferenza:} l’ontologia dichiara spesso tipi e classi in modo esplicito anche quando la stessa informazione potrebbe essere ricavata tramite reasoner — scelta che facilita l’uso “out-of-the-box” senza dover necessariamente eseguire un motore di inferenza.
  \item \textbf{Riuso di vocaboli esterni:} per unità di misura, quantità e aspetti commerciali l’ontologia fa riferimento a vocabolari consolidati (UN/ECE, \texttt{quot}, GS1, \texttt{schema.org}, SOSA/SSN), aumentando l’interoperabilità con altri dataset.
  \item \textbf{Estendibilità:} la struttura a moduli (stili, organizzazioni, packaging) facilita l’estensione futura: ad esempio aggiungendo dettagli su materiali dei contenitori, processi di produzione regionali, o collegamenti a dataset quali DBpedia/Wikidata.
\end{itemize}

\subsection{Esempio (Turtle)}
Un piccolo frammento esemplificativo in Turtle che mette insieme alcune delle entità e proprietà sopra descritte:
\begin{lstlisting}[language=turtle,caption={Esempio di istanza in Turtle}]
@prefix beer: <http://purl.org/ontology/beer/> .
@prefix xsd:  <http://www.w3.org/2001/XMLSchema#> .

beer:Guinness a beer:Beer ;
    beer:hasStyle beer:Stout ;
    beer:brewedBy beer:GuinnessBrewery ;
    beer:hasAlcoholContent "4.2"^^xsd:float ;
    beer:hasPackaging beer:GuinnessCan355 .

beer:GuinnessCan355 a beer:BeerProduct ;
    beer:netContent "355"^^xsd:integer ;
    beer:netContentUnit "ml" .
\end{lstlisting}

\subsection{Aspetti non coperti e sviluppi futuri}
L’attuale versione non copre ancora in modo esaustivo:
\begin{itemize}
  \item il materiale dei contenitori e il rivestimento interno delle lattine (rilevante per riciclo e interazioni chimiche);
  \item una tassonomia dettagliata degli ingredienti agricoli e delle loro varietà;
  \item il supporto esteso per packaging complessi e catalogazione logistica completa.
\end{itemize}
Questi temi sono candidati naturali per estensioni successive, mantenendo la compatibilità con i vocabolari esterni citati.


% % --- 4. Struttura concettuale dell'ontologia ---
% \section{Struttura concettuale dell’ontologia}
% L’ontologia definisce un insieme di concetti principali:

% \begin{itemize}
%   \item \textbf{Beer} – rappresenta il prodotto birra;
%   \item \textbf{Brewery} – il produttore o birrificio;
%   \item \textbf{BeerStyle} – la categoria o stile di birra (es. Lager, Ale);
%   \item \textbf{Ingredient} – gli elementi che compongono la birra;
%   \item \textbf{Country} – il luogo di origine.
% \end{itemize}

% \newpage
% \begin{figure}[!h]
%     \caption{Tipologie di birra}
%     \centering
%     \includegraphics[width=150mm]{beer_tpes.png}
%     \label{fig:label}
% \end{figure}
% \begin{figure}[!h]
%     \caption{Birrifici}
%     \centering
%     \includegraphics[width=150mm]{birrifici.png}
%     \label{fig:label}
% \end{figure}
% \newpage

% \newpage


% Le relazioni principali includono:
% \begin{itemize}
%   \item \texttt{brewedBy} – collega una birra al suo birrificio;
%   \item \texttt{hasStyle} – collega una birra al suo stile;
%   %\item \texttt{hasAlcoholContent} – descrive un valore numerico;
%   %\item \texttt{hasIngredient} – collega la birra agli ingredienti.
% \end{itemize}

\section{Casi reali e applicazioni della Beer Ontology}
La \textbf{Beer Ontology}\footnote{\url{https://rdf.ag/o/beer-en.html}} è una delle ontologie pubblicate nel portale \texttt{rdf.ag}, sviluppata da PSW Applied Research Inc. e rilasciata con licenza Creative Commons 3.0. 
Il suo scopo, dichiarato ufficialmente, è di ``fornire tracciabilità per birre e ingredienti, controllo di processo e identificazione degli stili in modo neutro rispetto alla lingua''.  
Essa fa parte di un ecosistema più ampio di ontologie per l’agroalimentare, integrato con standard industriali come \texttt{GS1}, \texttt{schema.org}, \texttt{SOSA/SSN} e \texttt{QUDT} per rappresentare unità di misura e packaging.

\subsection{Esempi ufficiali e struttura di riferimento}
La documentazione\footnote{\url{https://rdf.ag/o/beer-2024-02-21.html}} include esempi concreti di modellazione, come la \textit{ACME Brewery}, una birreria fittizia usata per illustrare l’uso di classi e proprietà in scenari realistici (\texttt{Beer}, \texttt{Brewery}, \texttt{BeerStyle}, \texttt{Packaging}).  
La versione più recente (v1.3, febbraio 2024) distingue chiaramente tra:
\begin{itemize}
  \item \textbf{Beer Styles} — organizzati in sottoclassi (Ale, Lager, ecc.) in base a ingredienti e processo di fermentazione;
  \item \textbf{Beer Making Organizations} — rappresentazioni di birrifici come entità produttive;
  \item \textbf{Packaging} — modellazione semantica di contenitori, dimensioni e unità, basata su GS1 e \texttt{quot ontology}.
\end{itemize}
Inoltre, l’ontologia distingue tra \textit{Beer} (sostanza) e \textit{Beer Product} (unità commerciale pronta per la vendita), importando concetti da \texttt{schema.org} e vocabolari per il commercio elettronico.

\subsection{Progetti e ricerche correlate}
La Beer Ontology è stata impiegata o presa a modello in diversi contesti di ricerca:
\begin{itemize}
  \item \textbf{Beverage Graph} — un knowledge graph per il settore delle bevande che integra birre, vini e succhi con dati di tracciabilità, stili e composizione.\footnote{Fois et al., ``Beverage Graph: Connecting Data about Consumable Liquids'', CEUR-WS Vol. 2969, 2021. \url{https://ceur-ws.org/Vol-2969/paper45-FoisShowCase.pdf}}
  \item \textbf{Fuzzy Beer Ontology} — estensione accademica usata in sistemi di raccomandazione basati su ontologie fuzzy e ragionamento incerto (progetti \textit{GimmeHop} e \textit{fuzzyDL}).\footnote{Straccia, U., Bobillo, F., ``FuzzyDL: Fuzzy Description Logic Reasoner'', CNR Pisa, 2021. \url{https://iris.cnr.it/bitstream/20.500.14243/485364/3/Straccia_IEEE_preprint.pdf}}
  \item \textbf{Zenodo Repository} — archivio pubblico con metadati e versioni storiche della Beer Ontology.\footnote{\url{https://zenodo.org/records/4672337}}
\end{itemize}

\subsection{Osservazioni finali}
L’ontologia della birra si distingue per la sua forte aderenza agli standard industriali e per l’obiettivo di connettere produttori, trasformatori e distributori in un unico grafo semantico.  
Nonostante la maggior parte dei casi d’uso sia ancora accademica o sperimentale, essa rappresenta un riferimento pratico per la modellazione semantica nel settore alimentare, in particolare come base per l’interoperabilità tra tracciabilità, etichettatura e commercio elettronico.



% --- 6. Valutazione critica ---
\section{Valutazione e punti di forza/debolezza}
\subsection*{Punti di forza}
\begin{itemize}
  \item Chiarezza nella struttura concettuale;
  \item Buon uso delle relazioni OWL;
  \item Facilità di estensione a nuovi domini.
\end{itemize}

\subsection*{Debolezze}
\begin{itemize}
  \item Ontologia datata e con copertura limitata;
  \item Mancanza di collegamenti a dataset esterni come DBpedia o Wikidata;
  \item Assenza di metadati sulle fonti.
\end{itemize}

\subsection*{Possibili estensioni}
\begin{itemize}
  \item Aggiunta di una tassonomia di ingredienti più dettagliata;
  \item Collegamento a dataset geografici per la localizzazione dei birrifici;
  \item Integrazione con dati di recensioni e valutazioni.
\end{itemize}

% --- 7. Conclusione ---
\section{Conclusione}
L’analisi della Beer Ontology ha permesso di comprendere l’importanza delle ontologie come strumento per rappresentare la conoscenza in modo formale e interoperabile.  
Nonostante la semplicità del dominio, il progetto illustra efficacemente i principi del Web Semantico e le potenzialità dei linguaggi RDF e OWL per l’integrazione e l’inferenza dei dati.

% % --- 8. Appendice (facoltativa) ---
% \appendix
% \section*{Appendice}
% \addcontentsline{toc}{section}{Appendice}
% \begin{itemize}
%   \item Screenshot della visualizzazione in Protégé.
%   \item Frammenti OWL o Turtle aggiuntivi.
%   \item Altri esempi di query SPARQL.
% \end{itemize}

% ================================
% Bibliografia
% ================================
\newpage
\addcontentsline{toc}{section}{Riferimenti bibliografici}
\bibliographystyle{plainurl}
\bibliography{references}


\end{document}
